%%%%%%%%%%%%%%%%%%%%%%%%%%%%%%%%%%%%%%%%%%%%%%%%%%%%%%%
%%% More SF & Feedback Text is commented out below %%%%
%%%%%%%%%%%%%%%%%%%%%%%%%%%%%%%%%%%%%%%%%%%%%%%%%%%%%%%

%So far, this combination of data exists mainly in the Milky Way and a few Local Group galaxies, and as a result, studies of cloud-scale star formation and feedback have concentrated mainly on this limited set of environments (e.g., Lee et al.\ 2016, Fukui \& Kawamura 2010): i.e., with the exception of the Galactic Center itself these are regions that are dominated by the atomic gas phase and have low stellar surface densities compared to normal, massive star-forming disks. Before JWST, the crucial IR-based diagnostics of star formation and feedback were limited by the coarse resolution of Spitzer and Herschel and so simply could not be obtained at “cloud scale” (10s of pc) resolution outside the Local Group.

%By pairing these estimates of the local star formation rate and cluster population with observations of the molecular cloud population, we can make a statistical measurement of the cloud life cycle. In the Milky Way, such observations suggest that most star formation is associated with only a few dozen (out of hundreds) of clouds, suggesting an evolutionary sequence in which massive star formation ``accelerates'' across a cloud’s lifetime and reaches a peak as it tears the cloud apart (Lee et al.\ 2016, Murray 2011). Such a violent, out-of-equilibrium cloud life cycle is at odds with some current theories (e.g., Krumholz \& Tan 2007, Krumholz \& McKee 2005). This scenario is essentially untested outside the Milky Way, where the light-of-sight confusion makes the interpretation challenging and unreliable. Pairing JWST-based SFRs with M51’s unique, high quality molecular catalog and our clean external view of the galaxy will allow strong tests of this and other models for the molecular life cycle, including the cloud lifetime and the time for feedback to disperse the cloud. This is only possible with a high resolution SFR tracer robust to dust (i.e., Ha and similar optical lines will not work).

%These goals pivot on using JWST to estimate the current rate of star formation and the physical state of the gas and dust on the scale of an individual cloud or cluster. They also depend on comparing to a well-understood molecular cloud population in a galaxy where the effects of galactic dynamics and local environment are also understood. In this regard, M51 is the ideal target. The molecular cloud population in the inner 9x7 kpc of the galaxy is uniquely well studied (Colombo et al.\ 2014, Hughes et al.\ 2013) thanks to the IRAM Large Program PAWS (Schinnerer et al.\ 2013). PAWS has been used to carry out preliminary work on cloud life times (Meidt et al.\ 2015) and testing turbulent models (Leroy et al.\ 2017), which show the feasibility (but also the pre-JWST limitations) of our science plan. More, the cluster population and galactic environment are also uniquely well-studied (This needs a few sentences, too). Cloud-scale star formation and feedback diagnostics from JWST are really the key missing ingredient to make a number of key measurements in the physics of cluster formation, star formation, and feedback.

%JWST has the opportunity to make major contributions to many of these, including: what is the life-cycle of a molecular cloud? Do stars form evenly through the life of a cloud or does star formation “accelerate,” destroying the cloud in the process (Murray 2011, Lee et al.\ 2016)? How do the properties of a molecular cloud relate to its instantaneous star formation rate? Do these results agree with turbulent star formation theory (e.g., Padoan et al.\ 2011, 2012), which make clear predictions (e.g., Federrath \& Klessen 2012) but are so far only very weakly tested by observations (e.g., Leroy et al.\ submitted)? How do the properties of molecular clouds relate to the stellar clusters that they form? Is there a link between the molecular cloud mass function, which varies within and among galaxies (e.g., Rosolowsky 2005, Colombo et al.\ 2014, Hughes et al.\ 2013, xx), and the output cluster population? And how strong is the feedback exerted by young stellar populations on the surrounding gas at any instant? What is the dominant feedback mechanism at each stage of a cloud’s evolution (e.g., Lopez et al.\ 2011, 2014)? These questions get at the physics underlying the scaling relations studied by Spitzer and Herschel. Thanks to the resolution of JWST, these can at last be addressed cloud-by-cloud in other galaxies.

%JWST will change this game completely. Its mid-IR continuum and IR recombination line imaging will provide extinction-robust star formation rate estimates at resolution <~ 1” ~ 40 pc at the distance of nearby massive star-forming galaxies like M51. This will yield, for the first time, extinction-robust star formation rates for each individual molecular cloud (rather than a large area) across a galaxy like M51. The same recombination lines will yield precise age and extinction estimates for young stellar clusters. Narrow band imaging of iron lines can reveal recent SNe, while H$_2$ line and PAH band ratios will diagnose the physical state of the gas and dust.

%Similar experiments can be used to evaluate the currently popular turbulent theories of star formation (e.g., Krumholz et al.\ 2012, 2005, Padoan et al.\ 2011, 2012). These theories posit a direct link between the density, Mach number, and self-gravity of a molecular cloud and its present day star formation rate (e.g., Federrath \& Klessen 2012). However, the lack of cloud-by-cloud SFR estimates has so far prevented detailed tests of these theories beyond the Milky Way and the Magellanic clouds, forcing a focus on population (and so time) averages (e.g., Leroy et al.\ 2017). These preliminary comparisons are already in some tension with the turbulent theories, highlighting the need for detailed, cloud-by-cloud studies to move the field ahead. Given that these turbulent theories underpin a wide range of numerical simulations and semi-analytic models, these test have large potential to impact our understanding of the evolution of galaxy disks across redshift.


%%% SPIRAL STRUCTURE text

%M51 contains a strong two-arm spiral with dark dust lanes along the arms and dust feathers extending into the interarm regions (La Vigne et al.\ 2006). It is one of the best-studied galaxies, with detailed observations including HST WFPC2 and NICMOS (Scoville et al.\ 2001, Calzetti et al.\ 2005), GALEX (Martin et al.\ 2005), Spitzer (Kennicutt et al.\ 2003), IRAM (Schuster et al.\ 2007; Chen et al.\ 2015), CARMA (Koda et al.\ 2009), VLA (Walter et al.\ 2008), PdBI (Schinnerer et al.\ 2013), and NOEMA (Chen et al.\ 2017) (it is too far north for ALMA). These observations have led to a large number of recent studies related to spiral structure and star formation, including an examination of pattern speeds (Meidt et al.\ 2008), streaming motions compared with GMC star formation (Meidt et al.\ 2015), cluster ages and offsets from arms (Schinnerer et al.\ 2017), and gas and arm offsets (Egusa et al.\ 2017). There have been many recent analytic (e.g., Meidt et al.\ 2008) and numerical simulations (e.g., Dobbs et al.\ 2010), including a high resolution one of a galaxy like M51 (Dobbs et al.\ 2017).  
 
%A key question that JWST can address concerns the link between spiral arms and star formation.  JWST will be able to observe for the first time the young massive stars and young star clusters that are still highly obscured by their natal clouds. Only the brightest of these sources have been found so far using Spitzer 3.6 mum images (0.75” pixels; Elmegreen et al.\ 2014). With much deeper IR observations at higher resolution, we will determine the environments of a large number of star-forming clouds. We will see if these clouds are still filaments aligned with the arms in the case of spiral arm shocks, or spurs, feathers, and shells downstream from the arms and therefore secondary, or clumps that condensed by gravitational instabilities or agglomerated in the shocks. We will also be able to see star formation in the interarm molecular clouds discovered by Koda et al.\ (2009).  These observations will allow the first direct determination of the star formation efficiencies in the arms and interarms to compare with models of spiral arm triggering and secondary processes.
 
%JWST imaging with NIRCam using the F360M filter (0.065” pixels) will show the embedded sources we wish to study. NIRCam imaging with the F187N filter will discern between massive stars and compact low-mass clusters by showing the obscured ionized gas at comparable resolution (0.032” pixels) through Pa a 1.87mm emission.

%%%% CLUSTERS text

%Understanding the lifecycle of star clusters is critical for understanding key processes in the universe, from star formation to feedback to galaxy evolution.  Possibly most stars form together in clusters, which in turn form in the densest parts---the clumps---of molecular clouds (e.g., Lada \& Lada 2003; McKee \& Ostriker 2007).  Even as they form, clusters begin losing mass due to feedback from massive stars, and eventually all clusters disrupt, returning their stars to the field within their host galaxy.

%JWST's capabilities open new observational windows into the star/cluster formation process. We will now be able to detect the very youngest embedded phase, a time when clusters are still enshrouded in their natal material.  Though stellar photospheric emission drops in the IR, the ionizing radiation conversely heats the surrounding dust to glow at 3--5$\micron$.  With our $\sim$10~pc maps of the molecular gas (Schinnerer et al.\ 2013) and the optically-selected $HST$ cluster catalog (Chandar et al.\ 2016), we will use the observed luminosity/mass function of embedded clusters in M51 to answer several basic questions about the star-formation process:  What is the typical SFE within a molecular cloud and does it depend strongly on mass?  How long do clusters remain in the embedded phase?  What fraction of stars form in clusters, and does this value approach 100\% during the youngest, embedded phase?  What fraction of embedded clusters are missed by shorter wavelength HST surveys?

%JWST Br$\alpha$ imaging combined with LEGUS H$\alpha$ imaging can yield accurate dust extinctions which in turn promote more accurate cluster ages and masses.  Existing Pa$\alpha$/H$\alpha$ will provide an empirical check on the new dust extinctions, at least for the more luminous clusters as the extant Pa$\alpha$ data are quite shallow.  The NIRCam F335M imaging will be used to identify clumps of PAHs and their spatial relationship to clusters, connecting the cluster environment to the survivability of PAH molecules.  Wolf Rayet stars (3--5~Myr; $>25 M_\odot$) will be identified through the presence of NIRSpec [CIV], HeI, HeII, and Br$\gamma$ emission.  Spectroscopic measurements of young massive star clusters will allow us to independently determine the age of the cluster via the these lines. In addition, WR stars are thought to play a role in the removal of the cluster's natal gas cloud. Thus, correlating the dust properties ($A_V$) for each cluster and their WR features will constrain the role of WR stars on feedback and the age at which gas is expelled from a star cluster. 

%Seven optically-identified young massive star clusters in M51 with WR signatures would be ripe targets (HeI 4686 A; Sokal+2016). 



% Text from SF & Feedback:  Stellar clusters are a fundamental output of the star formation process, but the origin of the fraction of stars in cluster and the star cluster population remain weakly measured. The mass function of clouds in M51 has been demonstrated to change as a function of dynamical environment, JWST will allow us to measure whether these different mass functions are imprinted into the output star cluster population. This needs a couple sentences from an expert explaining what was not possible with HST and a couple sentences outlining theoretical expectations. Note that Hughes et al.\ 2013b found that young stellar cluster (YSC) tracked well most of the GMC properties except for the slope of the mass functions which could be interpreted as being due to environment. Could JWST provide better measurements for stellar masses and a better age determination for clusters with ages < 10-50 Myr? It seems that this topic is closely related to one topic mentioned in ‘Spiral Structure & star formation’.





%In this regard, M51 is the ideal target. 
%The molecular cloud population in the inner 9$\times$7~kpc$^2$ is uniquely well studied (Colombo et al.\ 2014, Hughes et al.\ 2013) thanks to the IRAM Large Program PAWS \citep{schinnerer2013}.  
% SCOG: if we need to make cuts, we could consider shortening this dicussion of previous work
%These observations have led to a number of studies related to spiral structure and SF, including an examination of pattern speeds and streaming motions (Meidt et al.\ 2008, 2015), cluster ages and offsets from arms (Schinnerer et al.\ 2017), and gas and arm offsets (Egusa et al.\ 2017). There have been many recent analytic (e.g., Meidt et al.\ 2008) and numerical simulations of a galaxy like M51 (e.g., Dobbs et al.\ 2010, 2017). PAWS has been used to carry out preliminary work on cloud lifetimes (Meidt et al.\ 2015) and testing turbulent models (Leroy et al.\ 2017), which show the feasibility (but also the pre-JWST limitations) of this approach.  {\em JWST's access to the earliest stages of SF and to unique feedback diagnostics on cloud scales will provide the missing key ingredients to constrain the physics of cluster formation, SF, and feedback.}
% LKH A crucial step forward that JWST enables is evaluating the physical state of the gas as a function of cloud evolution. JWST offers a host of feedback diagnostics that can play key roles in this area: supernova remnants via [Fe{\small II}] or warm molecular gas via H$_2$ vibrational lines (and diagnostics of its heat source by comparison with PAH emission).



%Resolved mid-IR dust emission serves as a significant and mostly untapped tool for studying the evolution of galaxies. Of particular importance are PAH bands — skeletal vibrational emission features which arise from stochastically heated small aromatic grains at 3–17$\micron$. Typically 10\% and up to 25\% of the bolometric IR power in star forming galaxies is emitted in the PAH bands, dwarfing by a factor of 10-100x in luminosity any fine structure or recombination emission line a galaxy produces.  This fact alone makes PAH emission of significant potential diagnostic value.  The features are readily identifiable, and to first order follow a roughly uniform spectral template.  This has enabled PAH bands to be detected, with redshift sensitivity, in low-resolution spectra of galaxies up to z~4, and many plans are under development to extend this reach with blind redshift-sensitive PAH surveys to z=10 and beyond (SPICA, OST, etc.).   

%And yet, despite their ubiquity, surprisingly little is known about the PAH band carriers, with a variety of carbon-rich materials having been proposed since their discovery in the 1980’s.  In addition, we have only recently begun to explore how these small grains respond to changes in gas metal content, intensity and hardness of UV-optical starlight, the presence of AGN, the dominance of turbulent energy dissipation in shocks, etc.  And respond they do.  Local galaxy surveys with ISO and Spitzer have uncovered substantial variation in PAH band ratios, tied to the grain size distribution, ionization state, and relative abundance of these important grains.   Our goal is to develop these into full diagnostics of galaxy conditions.
%Small grains including PAHs regulate the flow of radiative energy through the ISM of galaxies through photoelectric heating.  This sets the temperature in neutral gas, impacting the process of star formation itself.  Yet the efficiency with which UV/Optical photons can be converted into thermal energy depends sensitively on the sizes and ionization states of PAH grains, information on which is currently limited due to the lack of coverage of PAH emission bands at the shortest wavelengths.

%PAH emission comprises $\sim$ 5\% of the entire, integrated gamma-ray to radio non-primordial radiative energy content of the Universe.  To realize the diagnostic potential of this energetically important emission, we need to resolve PAH emission in a wide variety of emitting environments in the local Universe.  Local galaxies present the ideal place to explore a diversity of environments — from intense PDRs to diffusely heated regions, to the inner 10s of pc in AGN centers; from super-solar metallicity to nearly pristine gas; from sites of ongoing star formation to regions dominated by the high intensity soft radiation fields of old stellar populations — it is through detailed, high resolution, high sensitivity, targeted narrow band continuum coverage and full spectral maps in carefully selected regions spanning the fullest range of environment that we can begin to harness the true potential of this relatively untapped emission.

%\noindent A few specific applications:

%\noindent $\bullet$ The size distribution of small grains is a critical ingredient in basic physical processes including the heating efficiency of neutral gas, the total carbon budget in grains, and the details of grain (re-)growth in dense clouds of gas.  The 3.3$\micron$ features are particularly important for understanding the sizes of the smallest grains, but have not been probed in nearby galaxies due to the 5.5$\micron$ lower cutoff of Spitzer/IRS.  How much of this could we do with NB imaging + Spitzer spectroscopy?

%\noindent $\bullet$ Ionization state: 17/3, or 11.3/3.3

%\noindent $\bullet$ Grain growth and destruction: direct tests inside and outside cold, dense clouds, H{\small II} regions, etc.  This is where the 5-10pc resolution comes into play.  Picking regions carefully is going to be entirely crucial.

%\noindent $\bullet$ Silicate absorption and emission, diagnostic of torus region in the Seyfert AGN nucleus (+ LINER in M51b if you choose to include it!).  Two classic weak AGN in one galaxy.

%\noindent $\bullet$ Aliphatic emission in the 3$\micron$ band.  The variation of PAH to aliphatic 3.4$\micron$ emission has not been explored except on global scales in high luminosity systems, and the ratio of aromatic (3.29$\micron$) to aliphatic (3.4$\micron$) vibrational structure is highly diagnostic of the structural state of the smallest grains (Joblin, 1996).

%\noindent $\bullet$ Ice and aliphatic grain absorption in the 3µm band.  The impressive spatial resolution provided by NIRCam/NIRSpec will likely isolate regions with dust columns ($A_V>3$) needed to appreciate 9.7 and 18$\micron$ silicate absorption.  With matched 1–28$\micron$ IFU data at these small scales, we will have a much better ability to probe global crystalline fraction, and to couple with multiple dust column estimates to uncover relations between optical and MIR opacity.  Full-up Ice vs. silicates: ice growth, etc. 
%Carbon budget issues?  How much carbon can there be in PAHs?  UV opacity like 2175 + PAH bands + ERE + CO + CI
